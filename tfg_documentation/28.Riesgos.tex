\chapter{Gestión de los riesgos}
Una vez descrita la planificación que se seguirá durante la realización del TFG queda analizar los posibles escenarios que puedan afectar de forma negativa al proyecto. Algunos de los riesgos menos graves ya han sido descritos en el apartado de Alcance así que no los trataremos. En este apartado trataremos únicamente los riesgos que puedan afectar de manera muy grave a la realización del trabajo.  

Hemos identificado 3 riesgos principales:
\begin{itemize}
    \item Ninguno de los entornos que hemos encontrado sirve como base para desarrollar el nuestro. En este caso, la tarea de estudio del entorno no tendría sentido y sería eliminada. Puesto que hacer un entorno desde 0 requiere mucho más tiempo, la asignación de tiempo de la tarea eliminada se añadiría a la implementación. Además la tarea de implementación se partiría en 2 nuevas tareas, implementación de las características básicas de un entorno e implementación de las características sociales del entorno. Dependiendo del tiempo necesario para la implementación, tareas de documentación podrían ver su tiempo reducido. Pero puesto que gracias a nuestra planificación contamos con varias semanas hasta la fecha final, esto debería ser una situación excepcional. Si esta contingencia sucediera es probable que la duración del proyecto se alargara.
    
    \item Debido a la situación compleja en la que nos encontramos después de largos meses de pandemia global, que parece que está llegando a su fin, pero podría recaer de nuevo, es posible que el proyecto pueda sufrir retrasos por motivos ajenos. En caso de ser por estos motivos excepcionales, se consultará al director del proyecto cuál es su consejo. Probablemente se restará tiempo a tareas de documentación y de investigación y se prolongaría la duración del proyecto. En caso de que esta situación sea demasiada extrema puede que sea necesario cancelar la realización del TFG. Aunque este escenario es muy poco probable, es necesario considerarlo.
    
    \item Otro riesgo considerable es que alguno de los roles que participa en el proyecto sufra de alguna enfermedad que pueda retrasar alguna de las tareas y por lo tanto el TFG. Aunque son pocas las probabilidades de que alguno de los roles implicados pase por esta situación, en caso de suceder es un riesgo importante para el proyecto. Si esto sucede, afortunadamente es posible aplazar la fecha límite del proyecto gracias a la planificación realizada.
\end{itemize}