\section{Tecnologías usadas durante el proyecto}

En el transcurso de este proyecto se han usado las tecnologías que listaremos a continuación. No haremos una explicación detallada de estas, ya que han sido explicadas en apartados anteriores o su uso es exclusivo en aspectos técnicos de la implementación. También cabe destacar que la mayoría de estas tecnologías fueron decididas en secciones posteriores del trabajo y no se conocían durante la fase de planificación inicial.

\begin{itemize}
    \item \textbf{\textit{Python}:} Lenguaje de programación usado para implementar la interfaz encargada de entrenar a los agentes de MARL.
    \item \textbf{\textit{Lua} y \textit{Love2D}:} Lenguaje de programación usado para implementar los cambios realizados en el entorno escogido.
    \item \textbf{Linux:} Sistema operativo encargado gestionar gran parte de la infraestructura de comunicación entre los procesos usado en el proyecto.
    \item \textbf{\textit{Xvfb}:} Programa encargado de crear un display virtual donde poder enviar el contenido gráfico del entorno escogido.
    \item \textbf{\textit{X11VNC}:} Programa encargado de gestionar la visualización de un virtual display para clientes de \textit{VNC}.
    \item \textbf{Visor de escritorios remotos:} Cliente de \textit{VNC} encargado de renderizar el contenido de un virtual display gestionado por X11VNC.
\end{itemize}

