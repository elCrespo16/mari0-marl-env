\section{Glosario}
\begin{itemize}
    \item Frameworks: Un framework es un marco o esquema de trabajo generalmente utilizado por programadores para realizar el desarrollo de software. Utilizar un framework permite agilizar los procesos de desarrollo, ya que evita tener que escribir código de forma repetitiva, asegura unas buenas prácticas y la consistencia del código.
    \item MMORPG: videojuego de rol multijugador masivo en línea
    \item Socket: Socket designa un concepto abstracto por el cual dos procesos (posiblemente situados en computadoras distintas) pueden intercambiar cualquier flujo de datos, generalmente de manera fiable y ordenada. Además es una estructura de datos que permite que dos máquinas se comuniquen entre ellas.
    \item Thread: En sistemas operativos, un hilo o hebra (del inglés thread), proceso ligero o subproceso es una secuencia de tareas encadenadas muy pequeña que puede ser ejecutada por un sistema operativo.
    \item Serializar: La serialización es el proceso de convertir un objeto en una secuencia de bytes para almacenarlo o transmitirlo a la memoria, a una base de datos o a un archivo. 
    \item Input: Entrada.
    \item Output: Salida.
    \item Open source: Software de código libre.
\end{itemize}