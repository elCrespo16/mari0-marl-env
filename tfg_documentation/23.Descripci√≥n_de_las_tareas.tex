\section{Descripción de las tareas}
Este proyecto estará divido en 2 fases diferenciadas, la fase de documentación y la fase de desarrollo del entorno. A continuación se realizará una descripción de cada tarea, se asignará unas horas de dedicación y se aportará una justificación a esta asignación. Más adelante, en la Tabla \ref {tab:planificacion-tareas} estará disponible un resumen donde se especificarán las estimaciones temporales, dependencias y recursos necesarios de forma más visual.

\subsection*{T1 Documentación}
Una parte muy importante de este proyecto consiste en la documentación generada, no solo para la memoria del proyecto, sino también para la posterior utilización de la herramienta desarrollada y la posible modificación del código. Por ello, podemos dividir las tareas de documentación en tres subsecciones: documentación para la memoria del proyecto, documentación para la utilización del entorno y documentación del código generado.

Gran parte de la documentación necesaria para la memoria forma parte de la realización de la asignatura obligatoria GEP. Las tareas realizadas en esta asignatura son fundamentales para el éxito del proyecto y representan una gran cantidad de horas, sobre todo el mes inicial y el periodo final del TFG.

\subsubsection*{T1.1 Contexto y Alcance}
En esta sección se establece la base del proyecto, se proporciona un contexto al trabajo y su alcance. Además se justifica la necesidad de este proyecto, que metodologías se usarán, posibles riesgos y objetivos y subobjetivos del proyecto. En resumen, de forma general se encamina el trabajo hacia la dirección correcta. El tiempo necesario para esta tarea han sido 20 horas. En esta dedicación de tiempo ya se ha tenido en cuenta la corrección necesaria después de la evaluación de los profesores.

\subsubsection*{T1.2 Planificación Temporal}
Se definen las fases del proyecto, la estimación temporal de este, las fechas clave y posibles dependencias entre tareas. Es una fase clave para organizar los diferentes elementos necesarios en la realización del trabajo. Se han invertido 10 horas en la planificación del proyecto.

\subsubsection*{T1.3 Presupuesto y sostenibilidad}
Esta tarea es similar a la tarea anterior, pero se trata principalmente de aspectos económicos. Se estima el presupuesto para el material y los recursos necesarios en la realización del proyecto. Además se tiene en cuenta el impacto que puede producir este trabajo en diferentes niveles. La dedicación prevista es de alrededor de 10 horas. Esto se debe a que la dedicación prevista por los profesores es de 9 horas, pero observando la tendencia de las tareas anteriores a desviarse 1 hora más aproximadamente, prevemos que esta tarea no será la excepción.

\subsubsection*{T1.4 Memoria final}
Esta sección es una de las partes clave del TFG, ya que servirá de referencia a futuros trabajos. Consiste en toda la documentación generada por las tareas anteriores y la generada por la parte técnica del proyecto. Puesto que se trata del documento principal del TFG y debe ser revisado y modificado diversas veces, se prevé una dedicación de 70 horas.

\subsubsection*{T1.4 Presentación}
Consiste en la preparación de la defensa del proyecto delante del tribunal que evaluará el trabajo. Esta fase se debe realizar después de la realización final de la memoria. Para esta tarea es necesaria la creación de cierto material audiovisual y diversos ensayos de la presentación. Por ello se estima una dedicación de 10 horas.

\subsubsection*{T1.5 Reuniones}
Consisten en las reuniones semanales realizadas con el equipo para revisar las tareas hechas cada semana, establecer las próximas, aclarar dudas, etc. La metodología ágil que usaremos para este proyecto recomienda sesiones cortas, pero puesto que el director y codirector tienen agendas complicadas, es posible que algunas sesiones se alarguen, consistiendo en varias reuniones en una. La previsión de tiempo estimada será de 6 horas.

\subsubsection*{T1.6 Documentación del entorno}
Puesto que el entorno se usará por otras personas, es necesario crear una documentación que explique las principales funcionalidades y su modo de empleo. La longitud de esta documentación depende de los aspectos complejos implementados en el entorno y este aspecto puede variar. Por ello, prevemos una dedicación de 40 horas. Además, esto deberá realizarse una vez completada la implementación del entorno y todas sus características.

\subsubsection*{T1.7 Documentación del código}
Otro de los aspectos importantes consistirá en la documentación del código, para que este pueda ser modificado por otras personas para añadir otras características. Esta tarea se puede realizar durante la implementación del código. Se estima una duración de 14 horas a esta tarea, puesto que el código generado puede ser extenso.

\subsection*{T2 Creación del entorno}

Esta sección forma la parte técnica del proyecto. Puesto que el proyecto es de investigación y desarrollo, las tareas consistirán en la búsqueda del entorno adecuado y las fases de la implementación de este.

\subsubsection*{T2.1 Búsqueda y valoración del estado del arte}
Parte inicial y crítica del proyecto donde se escogerá un entorno ya desarrollado para su posterior adaptación. Se inspeccionará el estado del arte en entornos multiagente complejos y en videojuegos y simuladores open source que se puedan adaptar a entornos multiagente, se compararán y se escogerá el que será usado en el proyecto. Puesto que se deben comparar las características de cada uno y dependiendo de esto el entorno final tendrá ciertas especificaciones u otras, se estima una duración de 60 horas.

\subsubsection*{T2.2 Aprendizaje del entorno}
Puesto que trabajaremos a partir de un entorno ya creado y deberemos modificarlo, una gran parte del tiempo estará dedicada a entender como está programado el entorno, sus funciones y capacidades. Y como es bien sabido, entender y modificar código de otras fuentes es un proceso complejo. Prevemos una dedicación de 40 horas.

\subsubsection*{T2.3 Diseño de las nuevas características}
Antes de empezar a programar, debemos tener claro el diseño de los elementos que añadiremos para que estos no entren en conflicto con los existentes y podamos tener una arquitectura fácilmente ampliable. Además, puesto que este campo es muy desconocido para el autor, en esta etapa se aprovechará para repasar conceptos clave. La dedicación aproximada será de 40 horas.

\subsubsection*{T2.4 Implementación de las nuevas características}
Esta será la fase en la que más tiempo se invertirá. Prevemos una dedicación de 80 horas, ya que se trata de una tarea compleja y hemos de tener en cuenta el tiempo necesario para solucionar los errores que vayan surgiendo.

\subsubsection*{T2.5 Testing y creación de los agentes}
Tal y como habíamos mencionado previamente, será necesario crear una serie de agentes que entrenen en el entorno creado. Esto servirá como testeo de que el entorno es funcional y realmente es capaz de entrenar agentes y como base de referencia para otros investigadores. Entrenar a un agente es una tarea compleja, pero nosotros lo haremos de forma básica, puesto que indagar más no forma parte de las competencias de este proyecto. Prevemos una dedicación de 50 horas.  

Puesto que la metodología Scrum se divide en sprints, explicaremos brevemente la partición de las tareas en los diferentes sprints.
\begin{itemize}
    \item \textbf{Sprint 1} Se realizarán las tareas de GEP (Contexto y Alcance, Planificación temporal y presupuesto).
    \item \textbf{Sprint 2} Se realizarán las tareas de la búsqueda del estado del arte y aprendizaje del entorno.
    \item \textbf{Sprint 3} Se realizarán las tareas de diseño, implementación del entorno, creación de los agentes y testing. Este sprint será más largo que los anteriores debido a que realizaremos 2 iteraciones sobre él, de esta manera evitando encontrarnos con fallos críticos al final del proyecto. Puesto que esta es la etapa más intensiva, es necesario tener como mínimo dos ciclos de diseño, desarrollo y testing. 
\end{itemize}
Cabe destacar que el resto de tareas de documentación no mencionadas en los sprints se irán realizando de forma paralela a todos los sprints.

Finalmente, la planificación final de todas las tareas estará demostrada de forma visual en el diagrama de Gantt \ref{fig:gantt}. Podemos observar que el camino crítico de las tareas consiste en las tareas de creación y desarrollo del entorno porque su ejecución es secuencial, por esto, si alguna de ellas sufre algún contratiempo grave, el resto de tareas y el proyecto se retrasarán. Por consiguiente, que hemos decidido partir la fase de diseño, implementación y testing en dos iteraciones, para evitar este problema sobre todo en las fases que son más vulnerables a contratiempos.
\begin{table}[t]
    \begin{center}
        \begin{tabular}{| c | l | c | c | c |}
        \hline
        \textbf{Código} & \textbf{Tarea} & \textbf{Tiempo} & \textbf{Dependencia} & \textbf{Recursos} \\ \hline
        \textbf{T1} & \textbf{Documentación} & \textbf{180 h} & & \\ \hline
        T1.1 & Contexto y Alcance & 20 h & & D, A, MAT, SOFT \\
        T1.2 & Planificación Temporal & 10 h & T1.1 & A, MAT, SOFT \\
        T1.3 & Presupuesto y sostenibilidad & 10 h & T1.1, T1.2 & A, MAT, SOFT \\
        T1.4 & Memoria final & 70 h &  & D, A, P, MAT, SOFT \\
        T1.5 & Presentación & 10 h & T1.4 & A, P, MAT, SOFT \\
        T1.6 & Reuniones & 6 h & & D, A, P, MAT \\ 
        T1.7 & Documentación del entorno & 40 h & T2.1 & A, P, MAT, SOFT \\ 
        T1.8 & Documentación del código & 14 h & T2.3 & A, P, MAT, SOFT \\ 
        \hline
        \textbf{T2} & \textbf{Creación del entorno} & \textbf{270 h} & &  \\ \hline
        T2.1 & Búsqueda y valoración del estado del arte & 60 h & & D, A, P, MAT \\
        T2.2 & Aprendizaje del entorno & 40 h & T2.1 & P, MAT, SOFT \\
        T2.3 & Diseño de las nuevas características & 40 h & T2.2 & P, MAT, SOFT \\
        T2.4 & Implementación de las nuevas características & 80 h & T2.3 & P, MAT, SOFT \\
        T2.5 & Testing y creación de los agentes & 50 h & T2.4 & P, MAT, SOFT \\
         \hline
         \textbf{Total} & & \textbf{450 h} & & \\
         \hline
        \end{tabular}
        \caption{Resumen detallado de las tareas que forman el proyecto [Elaboración propia]}
        \label{tab:planificacion-tareas}
    \end{center}
\end{table}
\begin{figure}
    \includegraphics[width=1\linewidth,keepaspectratio]{img/Planificación TFG_final.png}
    \caption{Diagrama de Gantt que representa la planificación temporal del proyecto. [Elaboración propia]}
     \label{fig:gantt}
\end{figure}

