\section{Identificación de leyes y regulaciones}

Durante la realización de este proyecto se usaron varias tecnologias. Repasaremos una a una estas técnologias para identificar las posibles regulaciones que puedan afectar al proyecto.

\begin{itemize}
    \item Python: La licencia de Python o Python Software Foundation Licence es una licencia de software libre permisiva, compatible con la licencia GPL (General Public Licence). Esta licencia permite hacer modificaciones del codigo fuetne  y la creación de trabajos derivados sin requerir que estas sean de codigo abierto. Esta licencia no puede afectar al proyecto.
    \item Librerias de Python (mss, PettingZoo): Estas librerias al igual que Python son de uso completamente libre y no pueden afectar al proyecto.
    \item Mari0 y mapas: Este juego y sus contenidos están bajo la licencia MIT. Esta licencia permite el uso del software sin ningun tipo de limitación. Esta licencia no afecta de ninguna forma al proyecto.
    \item Xvfb: Este programa al igual que Mari0 se encuentra bajo la licencia MIT.
    \item X11vnc: Este programa esta bajo la licencia GPL 2.0. Esta licencia permite el uso complemente libre, aún asi cualquier derivado de esta debe estar bajo la misma licencia o similar. Afortunadamente nuestro proyecto no genera no forma un derivado de esta tecnologia, tan solo se recomienda su uso, asi que esta tecnologia no afecta al proyecto.
    \item Vinagre: Esta tecnologia al igual que x11vnc esta bajo la licencia GPL y no afecta al proyecto.
\end{itemize}

La realización de este proyecto no esta actualmente afectada por ninguna regulación, ya que se trata de un campo realmente nuevo. Posiblemente en el futuro haya que tener en cuenta otros aspectos relacionados con la IA, pero en el momento no ha sido necesario.
