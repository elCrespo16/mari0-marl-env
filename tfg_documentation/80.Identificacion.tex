\chapter{Identificación de leyes y regulaciones}

Durante la realización de este proyecto se usaron varias tecnologías. Repasaremos una a una estas tecnologías para identificar las posibles regulaciones que puedan afectar al proyecto.

\begin{itemize}
    \item Python: La licencia de Python o Python Software Foundation Licence es una licencia de software libre permisiva, compatible con la licencia GPL (General Public Licence). Esta licencia permite hacer modificaciones del código fuente y la creación de trabajos derivados sin requerir que estas sean de código abierto. Esta licencia no puede afectar al proyecto.
    \item Librerías de Python (mss, PettingZoo): Estas librerías al igual que Python son de uso completamente libre y no pueden afectar al proyecto, en específico, todas las librerías usadas estaban bajo la licencia MIT.
    \item Mari0 y mapas: Este juego y sus contenidos están bajo la licencia MIT. Esta licencia permite el uso del software sin ningún tipo de limitación. Esta licencia no afecta de ninguna forma al proyecto.
    \item Xvfb: Este programa al igual que Mari0 se encuentra bajo la licencia MIT.
    \item X11vnc: Este programa esta bajo la licencia GPL 2.0. Esta licencia permite el uso complemente libre, aun así cualquier derivado de esta debe estar bajo la misma licencia o similar. Afortunadamente nuestro proyecto no genera no forma un derivado de esta tecnología, tan solo se recomienda su uso, así que esta tecnología no afecta al proyecto.
    \item Vinagre: Esta tecnología al igual que x11vnc esta bajo la licencia GPL y no afecta al proyecto.
\end{itemize}

La realización de este proyecto no esta actualmente afectada por ninguna regulación, ya que se trata de un campo realmente nuevo. Posiblemente en el futuro haya que tener en cuenta otros aspectos relacionados con la IA, pero en el momento no ha sido necesario.
