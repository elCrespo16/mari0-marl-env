\section{Sostenibilidad}
Una vez respondido la encuesta de sostenibilidad dirigida al ámbito universitario, tenemos una idea de nuestro conocimiento sobre la sostenibilidad y su relación con los campos TIC. Consideramos que el nivel teórico y práctico de nuestros conocimientos sobre la sostenibilidad son insuficientes comparados con el nivel esperado para un estudiante de ingeniería informática. Muchos de los conceptos presentados en la encuesta son completamente desconocidos para mí. Es posible que esto se deba a que al estudiar la rama de la computación, que está más enfocada al desarrollo y la investigación de recursos software donde no intervienen materias primas ni grandes generaciones de residuos, el campo de la sostenibilidad no se ha desarrollado de la forma adecuada.   

Por lo tanto, consideramos necesario aumentar de forma drástica nuestros esfuerzos en mejorar nuestros conocimientos en este campo tan importante. Afortunadamente, la realización de esta sección del TFG nos aportará cierto grado de conocimiento teórico y práctico. Gracias a las cuestiones planteadas en esta sección podremos obtener ciertas guías para continuar desarrollando estos conocimientos. No obstante, el trabajo realizado aquí será insuficiente para obtener el nivel deseado. Por ello, será necesario indagar en herramientas para medir la sostenibilidad de este proyecto. A continuación, realizaremos un análisis sobre la sostenibilidad de este TFG, empezando por la dimensión ambiental.

\subsection{Dimensión ambiental}
En primer lugar comenzaremos respondiendo las preguntas planteadas en la matriz de sostenibilidad  proporcionada por la asignatura de GEP. Estas preguntas servirán para cubrir los principales aspectos de la sostenibilidad en cada dimensión. Una vez contestadas estas preguntas, realizaremos un breve análisis de los riesgos que pueden surgir para cada dimensión.  

\textbf{¿Has estimado el impacto ambiental que tendrá la realización del proyecto?}  

El impacto ambiental generado en la realización de este trabajo consiste en: los ordenadores y otras máquinas, la energía utilizada y el papel necesario. Para la realización del trabajo serían necesarios 3 ordenadores personales para los tres roles que lo desarrollan y un servidor para realizar el entrenamiento de los agentes. Por lo tanto existe el coste de creación de estas herramientas. Puesto que en principio el hardware del servidor no es específico, podemos considerar que se trata de un cuarto ordenador. Según \cite{coste_ordenador} el coste de creación un ordenador se sitúa en 889 kg en emisiones de CO2, el impacto de nuestro proyecto consistiría de 3556 kg en emisiones de CO2. A estas emisiones es necesario sumarle el gasto energético en la realización del proyecto. Esta estimación se ha realizado en el apartado económico. Finalmente, el papel usado en este proyecto consistirá en la impresión de la memoria final, puesto que el resto de la documentación generada tiene formato electrónico.  

\textbf{¿Te has planteado minimizar el impacto, por ejemplo, reutilizando recursos?}

En primer lugar, el impacto medioambiental de los ordenadores se podría amortizar usando ordenadores de segunda mano en buenas condiciones y con eficiencias energéticas aceptables. Con esta medida también se conseguiría una reducción en el impacto económico. Desafortunadamente, no se puede disminuir el número de horas de trabajo y por lo tanto el gasto energético. Y de la misma forma, tampoco es posible disminuir el gasto de papel, puesto que ya se usa la mínima cantidad posible. 

Finalmente, de los posibles riesgos que podrían suceder que afectarán al ámbito medioambiental, hay que tener en cuenta la posibilidad de que el trabajo se retrase por cualquier posible contratiempo. Esto ocasionaría un mayor número de horas de trabajo y por lo tanto un mayor consumo energético. Y además, aunque poco probable, existe la posibilidad de que alguno de los ordenadores llegue al final de su vida útil y por ello haya que comprar otro. Ya hemos visto el impacto de esto en el medio ambiente.

\subsection{Dimensión económica}

En la sección de presupuestos se ha realizado una estimación de los costes del proyecto, teniendo en cuenta los recursos humanos y materiales.

\textbf{¿Cómo se resuelve actualmente el problema que quieres abordar(estado del arte)?}

El problema que quiero abordar actualmente, que es la falta de entornos multiagente con aspectos complejos, se resuelve usando los pocos ya existentes, porque crear uno nuevo requiere tiempo. Ese es precisamente el motivo por el cual se realizará este trabajo, para aportar al estado del arte un nuevo entorno con otras características.  

\textbf{¿En qué mejorará económicamente tu solución a las existentes?}
Añadir el entorno creado al estado del arte posibilitará la tarea de entrenamiento de agentes en entornos multiagentes complejos. Sin esta solución, es posible que se necesitará crear el entorno desde 0. Por esto, esta solución conseguirá ahorrar tiempo de desarrollo en otros posibles trabajos futuros. 

\subsection{Dimensión social}

\textbf{¿Qué crees que te va a aportar a nivel personal la realización de este proyecto?}
El entrenamiento por refuerzo es una de las áreas de la inteligencia artificial que más me han interesado, sobre todo por mi gran afición a los videojuegos. Realizar este trabajo me aportará las bases y el conocimiento de como funcionan estas técnicas de aprendizaje. Además, al tratarse del primer proyecto de proporciones considerables, me proporcionará las guías de como se deben realizar estos proyectos, ya sea en el ámbito empresarial o en el ámbito personal. Estos conocimientos no técnicos, que no se estudian directamente en las asignaturas de la universidad son valiosos y necesarios en el mundo laboral.  

\textbf{¿En qué mejorará socialmente (calidad de vida) tu solución a las existentes?}
La principal mejora que aportará este proyecto es que será posible el entrenamiento de agentes en entornos multiagente complejos, que de otra manera, sería necesario crear estos entornos, con el tiempo que esta tarea requiere, o simplemente no se podría realizar.  

\textbf{¿Existe una necesidad real del proyecto?}
Tal y como hemos mencionado en el apartado de contexto y justificación, el MARL es una rama que actualmente está en auge y son necesarios entornos complejos con comunicación entre agentes para investigar posibles soluciones a problemas de la vida real, donde estas características son habituales. Podemos concluir entonces que sí hay una necesidad real del proyecto. El proyecto es necesario para poder continuar con la investigación en esta rama de la inteligencia artificial.

