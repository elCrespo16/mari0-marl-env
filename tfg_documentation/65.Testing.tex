\section{Testing del entorno}

Para realizar el testing del entorno se utilizaron dos métodos distintos. El primer método consistía en probar a mano cada uno de las acciones por separado y comprobar que cada uno cumplía su función concreta. Además nos aseguramos que las recompensas se devolvían correctamente. Con este proceso asegurábamos que las acciones y las recompensas estuvieran bien diseñadas. Para realizar esta parte del testing se usó la tecnología \textit{VNC}. Esta tecnología permite que el contenido del \textit{Xvfb} pueda ser visualizado en la pantalla del ordenador en vez de solo en la memoria. En concreto se usaron los programas \textit{x11vnc} como servidor de \textit{VNC} y como cliente se usó \textit{Vinagre}. Estas tecnologías son más convenientes para el testing que usar la función \textit{Render} del entorno, ya que estas permiten interactuar directamente con el entorno usando el teclado y el ratón. Además, al usar esta tecnología puede verse todo el proceso del entorno, los resets, los tiempos de carga, etc. cosa que usando la función \textit{Render} no es  posible.

El segundo método consistía en usar la función de testing de \textit{PettingZoo}, esta se encarga de comprobar todas las funcionalidades de un entorno y además prueba la gran mayoría de las posibles combinaciones de las acciones. Una vez el entorno pasa estas pruebas está listo para ser utilizado en el entrenamiento de agentes. Esta es la última prueba, si se consigue entrenar agentes en el entorno entonces se puede considerar que el entorno es correcto.

Finalmente hicimos una tercera prueba que consistía en una mezcla de los dos métodos anteriores. Esta prueba la realizamos para testear que se pudiera jugar un humano además de varias inteligencias artificiales. Para ello ejecutamos la función de testing de \textit{PettingZoo} y activamos que se pudiera jugar un humano.
