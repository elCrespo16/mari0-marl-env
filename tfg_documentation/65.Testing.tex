\section{Testing del entorno}

Para realizar el testing del entorno se utilizaron dos metodos distintos. El primer metodo consisitia en probar a mano cada uno de las acciones por separado y comprobar que cada uno cumplia su funición concreta. Además nos aseguramos que las recompensas se devolvian correctamente. Con esto nos aseguramos de que las acciones y las recompensas estuvieran bien diseñadas. Para realizar esta parte del testing se uso la tecnologia VNC. Esta tecnologia permite que el contenido del Xvfb pueda ser visualizado en la pantalla. En concreto se usaron los programas x11vnc como servidor de VNC y como cliente se usó Vinagre. Estas tecnologias son más convenientes para el testing que usar la funcion Render() del entorno, ya que estas permiten interactuar directamente con el entorno usando el teclado y el raton. Además, al usar esta tecnologia puede verse todo el proceso del entorno, cosa que usando la función Render() no es completamente posible.

El segundo metodo consistia en usar la funcion de testing de PettingZoo, esta se encarga de comprobar todas las funcionalidades de un entorno y además prueba la gran mayoria de las posibles combinaciones de las acciones. Una vez el entorno pasa estas pruebas está listo para ser utilizado en el entrenamiento de agentes. Esta es la ultima prueba, si se consigue entrenar agentes en el entorno entonces se puede considerar que el entorno es correcto.

Finalmente hicimos una tercera prueba que consitia en una mezcla de los dos metodos anteriores. Esta prueba la realizamos para testear que se pudiera jugar un humano además de varias inteligencias artificiales. Para ello ejecutamos la funcion de testing de PettingZoo y activamos que se pudiera jugar un humano.
