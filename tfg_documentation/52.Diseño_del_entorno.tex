\section{Diseño del entorno}



\subsection{API de PettingZoo}

Como ya habiamos mencionado en el apartado del Estado del arte, PettingZoo es una librería que ofrece una gran cantidad de entornos multiagente. Una de las ventajas de esta librería, es que ofrece una estructura de creación de entornos muy clara. Esta estructura es muy similar a la estructura de creación de entornos de OpenAI que ya habiamos comentado anteriormente. Seguir esta estructura en un entorno facilita el entrenamiento de agentes, ya que no es necesario modificar el codigo demasiado si la mayoria de entornos siguen el mismo patron. Por esto, tal y como habiamos planteado en el apartado de Alcance, usaremos la estructura de PettingZoo para nuestro entorno.

En primer lugar, para diseñar un entorno es necesario que tipo de entorno será. Para esto PettingZoo provee de diferntes clases para diferentes tipos de entorno. Existen varios tipos de entorno multiagente, por ejemplo: entornos por turnos, entornos paralelos, etc. En nuestro caso, los agentes que actuaran en el juego lo harán a la vez, por lo tanto, la clase más apropiada será ParallelEnv, diseñada para entornos donde ambos agentes actuan en paralelo.

Una vez elegida la clase a partir de la cual crearemos el entorno, faltaría diseñar unas funciones y atributos especificos, que gracias a la estructura de PettingZoo siempre son los mismos. Las funciones y atributos que deberemos implementar son las siguientes:

\begin{itemize}
    \item Atributo observation\_ space.
    \item Atributo action\_ space.
    \item Atributo possible\_ agents.
    \item Función Render.
    \item Función Close.
    \item Función Reset.
    \item Función Step.
\end{itemize}

En los siguientes apartados realizaremos una breve explicacion cual es el objetivo de cada uno de estos metodos o atributos.

\subsubsection*{Atributo observation\_ space}

Este atributo sirve para definir que forma tendrán las observaciones que proporciona el entorno. Para definir este atributo se usa la API de OpenAI, en particular el modulo Spaces. En nuestro caso, como explicaremos más adelante, se usará una imagen RGB de dimensiones 600 x 800. Esta se representa usando la API de OpenAI como la clase Box.

\subsubsection*{Atributo action\_ space}

Este atributo sirve para definir que forma deben tener las acciones tomadas por el agente. Al igual que con el atributo observation\_ space, se define usando la API de OpenAI. En nuestro caso, como explicaremos más tarde, usaremos una lista de diferentes valores donde cada posicion de la lista codificará el uso de las diferentes acciones que puede realizar el agente en el entorno. Esta lista se representa usando la API de OpenAI como la clase MultiDiscrete.

\subsubsection*{Atributo possible\_ agents}

Este atributo define los posibles agentes que puede tener el entorno. En nuestro caso especifico, puede tener 2 jugadores.

\subsubsection*{Función Render}

Esta funcion se encarga de mostrar de forma grafica lo que esta sucediendo en el entorno. Esto puede ser mostrando texto por la consola, creando un display de imagenes, etc. En nuestro caso, se abrirá una ventana del modulo Pygame \cite {pygame} con la información de la ultima captura de pantalla realizada.

\subsubsection*{Función Close}

Esta funcion se encarga de cerrar el entorno de forma segura, destruyendo los elementos que se hayan de destruir. En nuestro caso, es necerio cerrar el modulo mss \cite {mss}, cerrar el thread de communicación, destruir el proceso de Xvfb \cite {xvfb} y el proceso de Mari0.

\subsubsection*{Función Reset}

Esta funcion se encarga de rehacer el entorno de forma que sea seguro llamar a los metodos Step y Render sin problemas. Además debe comenzar el entorno desde el principio. En nuestro caso simplemente ejecutamos la funcion de Reset implementada en el juego.

\subsubsection*{Función Step}

Esta funcion se encarga de recibir las acciones de los agentes, ejecutarlas y devolver la información del entorno. En particular debe devolver para cada agente la observacion del entorno, la recompensa obtenida por las acciones realizadas, si el entorno se ha finalizado y alguna infomación adicional.


Una vez explicado la API de PettingZoo que utilizaremos para crear nuestro entorno, pasaremos a explicar detalladamente Mari0 y sus componentes. Para facilitar las futuras explicaciones, explicaremos los elementos básicos que componen este videojuego, asi como su objetivo y sus mecanicas principales.

\subsection{Mari0: Mario Bros con portales}

Tal y como hemos explicado anteriormente, este videojuego es la combinación entre los miticos Mario Bros y P0rtal. Al ser una combinación entre estos videojuegos comparte muchos de sus elementos, entre ellos su objetivo. El objetivo de estos videojuegos es llegar al final de cada nivel. En cada nivel se proponen diferentes retos y usando los elementos del entorno disponibles y las mecanicas propias del juego, se deben resolver estos retos. El juego está limitado a 4 jugadores como máximo.

\subsubsection{Mecanicas}

Las mecanicas pricipales de este juego son el movimiento basico del Mario y la pistola de portales. El movimiento basico consiste en el movimiento a izquierda o derecha y el salto. La pistola de portales es algo más compleja. 

\subsubsection*{Movimiento y acciones}

El movimiento y las acciones basicas de este juego son más variadas que en juegos similares. Algunas de estas acciones no se han implementado en el entorno porque añadian demasiada complejidad al entorno o simplemente porque no añadian ninguna ventaja. Las pricipales acciones que se pueden realizar son:

\begin{itemize}
    \item Movimiento: Moverse a la derecha o izquierda.
    \item Sprint: cuando se esta moviendo hacia alguna direccion es posible esprintar. Esta accion no está implementada en el entorno.
    \item Salto alto: Este es el salto en el cual se llega a maxima altura. Este es el unico tipo de salto que esta implementado en el entorno.
    \item Salto bajo: Si una vez presionado el boton de salto, se deja de presionar, el jugador realiza un salto de menor altura. Este tipo de salto no está implementado.
    \item Usar objeto: Esta accion usa el objeto que esta más cerca del jugador. Se usa principalmente para agarrar cajas y accionar pulsadores.
    \item Ataque: Esta accion solo puede usarse cuando el jugador esta en el estado de Mario de fuego. El jugador suelta una bolita de fuego que rebota y daña a los enemigos que golpee. Esta accion no esta implementada en el entorno.
    \item Portal 1 y 2: Con esta accion el jugador dispara su portal 1 o 2 en un angulo desde su posición.
    \item Recargar portales: Con esta accion el jugador retira todos los portales que haya colocado.
\end{itemize}

\subsubsection*{Portales}

Cada jugador dispone de una pistola con dos tipos de portales, el portal izquierdo y el derecho o portales 1 y 2, cada uno representado con un color distinto. Estos portales se pueden atravesar por los todos los jugadores y por otros elementos como cajas y laseres. En la figura \ref {fig:portal} podemos ver mejor el funcionamiento de un portal.

\begin{figure}[h]
    \centering
    \includegraphics[width=0.9\textwidth]{img/portal-function.png}
    \caption{Funcionamiento de un portal [Elaboración Propia]}
    \label{fig:portal}
\end{figure}

Si el jugador que aparece en la figura atraviesa uno de los lados del portal, aparecerá en el otro lado del portal manteniendo la dirección del movimiento y su velocidad. Extrapolando este conocimiento, cuando un jugador atraviesa un portal, la magnitud de su velocidad se mantendrá intacta al salir por el otro lado del portal, mientras que su dirección estará dictada por la dirección en la que está enfocada el portal de salida.

Aunque los portales son una gran herramienta, tambien tienen sus limitaciones. Las principales limitaciones que hemos encontrado son:

\begin{itemize}
    \item Los portales solo pueden colocarse sobre superficies de color grisaceo como la que podemos ver en la figura anterior.
    \item No es posible disparar proyectiles de portales a traves de un portal abierto.
    \item Cada jugador solo puede tener colocado unicamente 1 portal de cada tipo a la vez. Además, los proyectiles de los portales viajan unicamente en linea recta.
    \item Existen algunos elementos que no permiten que los proyectiles de portales los atraviesen y cuando un usuario pasa a traves de estos, todos los portales colocados por este jugador se eliminan. Adicionalmente, el jugador dispone de una tecla para eliminar todos los portales colocados voluntariamente.
\end{itemize}


\subsubsection{Elementos del entorno}

\subsubsection*{Laseres azules}

Estos laseres son tangibles por los jugadores y por lo tanto pueden servir como platarma o como obstaculo. Adicionalmente, estos laseres pueden atravesar portales dandoles mucha más utilidad. Finalmente, los proyectiles de portales pueden atravesar estos laseres sin sufrir ninguna consecuencia. En la figura \ref {fig:laser-azul} podemos ver un ejemplo de estos laseres.

\begin{figure}[h]
    \centering
    \includegraphics[width=0.1\textwidth]{img/laser-azul.png}
    \caption{Laser azul de Mari0 [Elaboración Propia]}
    \label{fig:laser-azul}
\end{figure}

\subsubsection*{Laseres anti-portales}

Estos laseres son elementos que los jugadores pueden atravesar pero con ciertas consecuencias. En primer lugar, los proyectiles de portales no pueden atravesar estos laseres. Adicionalmente, cuando un jugador los atraviesa, eliminan todos los portales colocados por este jugador. Este laser es muy util en la creacion de niveles y sobretodo para prevenir situaciones extrañas donde un jugador sea teletrasportado a una parte del nivel que ya se haya superado.

\begin{figure}[h]
    \centering
    \includegraphics[width=0.1\textwidth]{img/laser-antiportal.png}
    \caption{Laser antiportal de Mari0 [Elaboración Propia]}
    \label{fig:antiportal}
\end{figure}

\subsubsection*{Elementos accionadores}

Dentro del entorno existen varios elementos que se pueden accionar, desencadenando así una accion. Los elementos más comunes son botones y pulsadores. Los primeros se accionan cuando tienen un jugador o una caja encima de ellos mientras que los ultimos se accionan mediante la tecla de uso del jugador. Ambos disponen a su vez de una linea que sirve para indicar que elemento accionan. Podemos ver ejemplos de estos elementos en las figura \ref {fig:boton}.

\begin{figure}[h]
    \centering
    \includegraphics[width=0.1\textwidth]{img/boton.png}
    \includegraphics[width=0.05\textwidth]{img/pulsador.png}
    \caption{Boton  y pulsador de Mari0 [Elaboración Propia]}
    \label{fig:boton}
\end{figure}

Además de estos elementos que hemos explicado, existen muchos otros como los geles, cajas, laseres rojos, palancas de salto, laseres anti-gravitatorios etc. El objetivo de este apartado no es explicar todos los elementos que existen si no solo los pricipales. Con los elementos que hemos explicado hasta ahora es suficiente para continuar nuesta explicación del diseño del entorno. 



Una vez explicado los elementos del videojuegos, pasaremos a explicar las diferentes tareas que tuvimos que realizar para el funcionamiento del entorno. En primer lugar, era necesario realizar un diseño que nos permitiera cumplir los objetivos propuestos. La primera tarea a realizar fue encontrar un mapa para 2 jugadores como mínimo.

\subsection{Mapa cooperativo}
Mari0 es un juego opensource desarrollado por la comunidad. Este juego obtuvo mucha popularidad en los años 2012 a 2015. Durante esta epoca se creo un foro sobre el juego donde cualquier diseñador interesado podía enviar una copia de su mapa para que el resto de los jugadores del juego pudieran valorarlo. El juego ya trae varios mapas creados por los creadores del juego. Todos estos mapas pueden jugarse con más de 1 jugador, pero solo es necesario 1 jugador para completarlos. Aún así, ninguno de estos es estaba completamente diseñado para un minimo 2 o más jugadores.

En nuestro caso es necesario un mapa diseñado para 2 jugadores como minimo. Por ello teniamos 2 opciones. La primera opcion consistía en implementar el mapa nosotros mismos. Esta tarea no es demasiado dificil ya que el juego trae consigo un editor de niveles muy intuitivo y facil de usar. Aún así, esta tarea no estaba prevista y aún ser sencilla, es necesario un tiempo para desarrollar y testear el mapa. La otra opcion consistia en realizar una busqueda por el foro \cite{mari0-forum} con el objetivo de encontrar algún mapa que cumpliera estas características. 

Decidimos buscar en el foro y en caso de encontrar un mapa, analizar si era viable usarlo como el mapa pricipal del entorno. En caso de no encontrar ninguno viable, habría que implementarlo desde 0. Para facilitar la busqueda, se contacto con el usuario HugoBDesigner, un diseñador muy popular de mapas de Mari0, ya que este se dedicaba a realizar valoraciónes de los mapas que le enviaban otros usuarios y colaboraba activamente con los creadores del videojuego. Este diseñador colaboró con la busqueda aportadonos enlaces a 3 mapas cooperativos que él había jugado durante la epoca del 2012. De estos 3 enlaces, solo 1 permanecía activo. Este mapa llamado Bowser Cooperative Testing Initiative fue el que usamos para como mapa principal para el entorno. El mapa puede descargarse de \cite {mari0-mapa} gracias al diseñador Pixel Worker. 

Para comprobar que el entorno no puede ser resuleto por un solo agente usaremos la figura \ref {fig:mapa} que es el principio del primer nivel del mapa.

\begin{figure}[ht]
    \centering
    \includegraphics[width=0.9\textwidth]{img/mario-1-level.png}
    \caption{Nivel del mapa Bowser Cooperative Testing Initiative \cite {mari0-mapa}}
    \label{fig:mapa}
\end{figure}

La forma de resolver este nivel es la siguiente:
\begin{itemize}
    \item Paso 1: En primer lugar, uno de los dos jugadores, de ahora en adelante jugador 1, debe colocar un portal en cada uno de los circulos rojos indicados en la figura. Con este movimiento se consigue que el jugador 1 pase a la sección superior.
    \item Paso 2: El jugador 2 debe colocarse donde se encuentra la X indicada en la figura. Desde esta posicion debe disparar dos portales a los circulos verdes indicados en la figura. Con este paso el jugador 1, que se encuentra en la sección superior, será capaz de atravesar estos portales y pasar fuera del cuadrilatero inicial.
    \item Paso 3: Una vez el jugador 1 se encuentra fuera del cuadrialtero, el jugador 2 debera realizar las misma acciónes que el jugador 1 en el primer paso.
    \item Paso 4: Mientras eso ocurre, el jugador 1 deberá colocar sus portales donde los colocó el jugador 2 en el paso 2.
    \item Si se han realizado correctamente todos los pasos, ambos jugadores deberian haber salido del cuadrialtero inicial.
\end{itemize}

Es facilmente visible que un solo jugador no puede completar este primer nivel. Esto se debe a que desde la sección superior es imposible colocar los portales donde debe para salir del cuadrilatero. El resto de niveles del mapa incluyen este tipo de puzles los cuales, al igual que este, necesitan como minimo 2 agentes para ser resueltos.

Una vez encontado el mapa, debiamos decidir como estarían codificadas las acciones de los agentes y las observaciones usando la API de OpenAI.

\subsection{Acciones}

En el apartado de \textit{Mari0}: \textit{Mario Bros} con portales explicamos las diferentes acciones que un jugador podía ejecutar en el juego base. Aunque estas acciones tienen sentido cuando el juego se utiliza por personas, algunas de estas no aportan ningún valor para el entorno. 

Así que decidimos filtrar cuáles acciones de todas las posibles se quedarían en el entorno final. Para seleccionar estas acciones se analizó cuál era el conjunto mínimo de acciones necesarias para resolver el mapa del entorno. Estas acciones correspondían a moverse a izquierda y derecha, salto alto, usar objeto, disparar portales 1 y 2 y recargar portales. La acción de salto bajo no añade ninguna mejora al set de movimiento, ya que no hay ninguna ventaja en hacer un salto de menor tamaño. Y la acción de atacar tampoco se utiliza en el mapa, ya que el jugador nunca llega al estado de Mario de fuego.

La siguiente tarea era decidir como se codificarían estas operaciones en la API de OpenAI. Todas las operaciones deberían tener la posibilidad de no ser ejecutadas. Nos dimos cuenta de que era posible codificar todas las acciones usando números enteros. La codificación de las acciones fue la siguiente:

\begin{itemize}
	\item Si el valor de la acción es 0 significa que no debe ejecutarse.
	\item Para acciones de una sola variante como Saltar y Recargar portales, el valor 1 significa ejecutar esa acción.
 	\item Para la acción de Usar objeto, la orientación del personaje importa, por lo tanto era necesario dar la posibilidad de usar un objeto a la derecha o a la izquierda del jugador. Por lo tanto el valor 1 simboliza coger un objeto a la izquierda del jugador y 2 un objeto a la derecha.
	\item Para la acción de movimiento a ambos lados el valor 1 codifica movimiento a izquierda y el valor 2 a derecha.
	\item Para las acciones de disparar portal 1 y disparar portal 2, cualquier valor diferente a 0 indica el ángulo con respecto al jugador en el que se debe disparar el portal.
\end{itemize}

Cada acción tiene una posición específica en la lista. Por ejemplo, la lista de acciones podría verse como en la Tabla \ref {tab:accion}. Esta lista codificaría que el jugador se está moviendo para la izquierda mientras realiza un salto y dispara el portal 1 con un ángulo de 15 grados.
\begin{table}[h]
	\begin{center}
		\begin{tabular}{| l | l | l | l | l | l |}
			\hline
			\textbf{Movimiento} & \textbf{Salto} & \textbf{Usar} & \textbf{Recargar} & \textbf{Portal 1} & \textbf{Portal 2} \\ \hline
			1                   & 1              & 0             & 0                 & 15                & 0                 \\ \hline
		\end{tabular}
		\caption{Ejemplo de set de acciones de un agente [Elaboración propia]}
		\label{tab:accion}
	\end{center}
\end{table}


En otros juegos similares se suele escoger una codificación más simple, donde todas las acciones simples se juntan en una única posición. En nuestro caso no decidimos usar esta codificación, ya que queríamos permitir que las acciones se ejecutaran al mismo tiempo. Aun así, la acción de moverse hacia derecha o izquierda se ha unido en una sola posición, ya que son acciones complementarias.

\subsection{Tipo de observaciones}
La siguiente tarea a realizar era decidir que tipo de obsevaciones extraeriamos del entorno. Tal y como habiamos explicado en el apartado de Definición de conceptos, los agentes necesitan conocer el estado del entorno para poder aprender y tomar sus acciones. Para ello es necesario obtener una observación del juego. 

Existen dos principales formas de obtener el estado del juego: Serializando alguna clase que contenga los pricipales elementos del juego o simplemente capturar la imagen producida por el videojuego. La primera alternativa se suele usar en entornos más simples donde los elementos importantes para entrenar son pocos y facilmente serializables. Un ejemplo de este tipo de entornos puede ser el entorno Level-Based Foraging visto en el apartado Estado del arte. En este entorno simplemente se pasa una cuadricula con los elementos que hay en cada una de ellas. Podemos ver la representación del entorno en la figura \ref {fig:foraging-2}.

\begin{figure}[h]
    \centering
    \includegraphics[width=0.3\textwidth]{img/level-base.png}
    \caption{Entorno gráfico de Level-Based Foraging \cite {env-list}}
    \label{fig:foraging-2}
\end{figure}

La segunda alternativa suele ser usada en entornos más complejos donde serializar todos los elementos es realmente complejo. Un ejemplo que utiliza esta alternativa es el entorno MALMÖ.

Nuestro entorno no es lo suficientemente simple para poder usar la primera opcion. Esto se debe los elementos importantes para el entrenamiento de un agente en nuestro entorno son muchos, muy variados y dificilmente serializables. Tomemos por ejemplo la figura \ref {fig:observartion} que forma parte de uno de los primeros niveles del mapa. 

\begin{figure}[h]
    \centering
    \includegraphics[width=0.9\textwidth]{img/BCTI_observation.png}
    \caption{Nivel del mapa Bowser Cooperative Testing Initiative \cite {mari0-mapa}}
    \label{fig:observartion}
\end{figure}

Esta sería la lista de elementos que deberíamos serializar para el entorno en caso de usar la primera alternativa:

\begin{itemize}
    \item Todos los bloques donde los jugadores pueden golpearse, incluyendo el techo del mapa. Esto es asi ya que gracias a los portales estos bloques pueden ser obstaculos o tener un papel relevante en el nivel.
    \item La posicion de los jugadores.
    \item La posicion de botones, pulsadores, palancas de salto, dispensadores de gel, puertas y dispensadores de cajas. Y además incluir la relación de que elementos accionan cada boton, pulsador etc.
    \item Todos los bloques donde los jugadores pueden colocar portales.
    \item Lasers azules, lasers rojos y lasers de destruccion de portales.
    \item Partes acuaticas o de acido.
    \item Enemigos y elementos dañinos.
\end{itemize}

Serializar esto es una tarea complicada y más aun teniendo en cuenta que los modulos del sistema no estan diseñados para esto. Si el juego se hubiera diseñado con la idea de pasar todos estos elementos a otro programa, esta sería una opcion viable. Pero en nuestro caso, esta opcion requeriría reconstuir practicamente todo el juego, algo impracticable. De modo que la unica opcion que queda es capturar las imagenes producidas por el videojuego y enviarlas a los agentes. Para esto, se decidio utilizar la resolución base del videojuego (600 x 800) y obtener sus imagenes en formato RGB. Por lo tanto, el tamaño de la observación sería 800 * 600 * 3 * 255.

Una vez diseñada la codificacion de las observaciones y las acciones, la siguiente tarea consisitia en diseñar un sistema para capturar el output del juego y pasarlo a los agentes.



