\chapter{Alcance}
Por lo presentado en los apartados previos, \textbf{el objetivo principal del proyecto es el desarrollo de un entorno multiagente con elementos sociales complejos}. La consecución de esta meta aportará a los investigadores de MARL un nuevo entorno donde investigar comportamientos de agentes en diversas situaciones. Para conseguir este objetivo, nos definiremos una serie de objetivos genéricos.

En primer lugar, debemos escoger adecuadamente el entorno que adaptaremos para nuestro objetivo. Tal y como hemos visto existen varios posibles candidatos con sus ventajas y desventajas. Esta elección es clave, ya que marcará en gran parte el desarrollo del resto del proyecto. 

En segundo lugar, debemos decidir que aspectos sociales queremos implementar en este entorno. Esto depende en gran manera del entorno de partida, por ello no es posible hacerlo antes. Ejemplos de estos aspectos podrían ser: comunicación directa entre agentes mediante una serie de mensajes predefinidos, un sistema de comercio de recursos entre los agentes, un sistema de roles específicos para cada agente que obligue a su colaboración, etc. 

Finalmente deberemos modificar el entorno para añadir estos aspectos que hemos decidido implementar y crear unos agentes que sirvan como modelo de referencia. Una vez realizados estos objetivos genéricos, el trabajo estaría prácticamente completo. 

Además, hemos identificado una serie de subobjetivos que garantizarán que el proyecto sea consistente y esté completo:

\begin{itemize}
    \item Inspeccionar el estado del arte de en entornos multiagente. 
    \item Definir unas métricas cualitativas que permitan valorar los posibles entornos y sus características.
    \item Entender la construcción del entorno para utilizar todas las oportunidades que nos pueda ofrecer modificando el menor código posible.
    \item Entrenar una serie de agentes para comprobar el funcionamiento del entorno. 
    \item Garantizar que el entorno que creemos o modifiquemos use la estructura de OpenAI / PettingZoo.
\end{itemize}
 
 \section{Requisitos}

Hemos identificado algunas necesidades para garantizar la calidad del proyecto:
\begin{itemize}
    \item Es necesario crear una documentación útil en el uso del entorno, de manera que sea sencillo de implementar en otras investigaciones.
    \item Es recomendable el uso de buenas prácticas de programación para conseguir código legible y lo más simple posible. 
    \item Para garantizar el rigor del entorno, es necesario entrenar un agente para que futuras investigaciones tengan un modelo de referencia.
    \item Los entornos considerados como candidatos deben ser recientes y obviamente debe estar disponible su código fuente. 
    \item Es muy recomendable usar estructuras de datos y algoritmos eficientes para que la tarea de entrenamiento de agentes sea eficaz, debido a que será ejecutada una gran cantidad de veces.
\end{itemize}


\section{Riesgos}
Para un correcto desarrollo de todas las etapas del proyecto es necesario considerar posibles riesgos que puedan surgir. Los principales riesgos que hemos encontrado son los siguientes:
\begin{itemize}
    \item Ninguno de los entornos encontrados cumple con las expectativas. \textbf{Solución}: Crearemos el entorno desde 0 usando las herramientas que mejor nos convengan.
    \item El tiempo de crear un nuevo entorno es superior al estimado. \textbf{Solución}: Reduciremos los elementos a implementar quedándonos solo con los aspectos más clave o en caso de ser posible, extenderemos la fecha de finalización del proyecto.
    \item El ordenador donde se está desarrollando el proyecto deja de funcionar. \textbf{Solución}: Tenemos otros ordenadores disponibles y en caso de necesitar comprar otro, se tendría en cuenta en los presupuestos. 
    \item El proyecto necesita una capacidad de cómputo superior a la que tenemos. \textbf{Solución}: El grupo de trabajo del director del proyecto puede facilitarnos más capacidad de cómputo.
    \item Falta de coordinación entre el autor y el director del proyecto. Puesto que ambas personas tienen otros deberes que llevar a cabo, es posible que sea complicado mantener una buena coordinación. \textbf{Solución}: El autor se coordinará con el director y el codirector del proyecto de la mejor manera posible. 
\end{itemize}
    