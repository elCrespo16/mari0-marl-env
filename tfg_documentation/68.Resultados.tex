\section{Resultados}

Los resultados obtenidos despues de haber realizado la parte técnica del proyecto son los siguientes:
\begin{itemize}
    \item Hemos inspeccionado exhaustivamente el estado del arte en entornos multiagente y plataformas opensource que puedieran servir como posibles entornos multiagente. Más adelante conseguimos definir unas metricas cualitativas para evaluar los entornos que habiamos encontrado.
    \item Hemos desarrollado un entorno con capacidad de entrenar a 4 agentes como maximo. Este entorno obliga a la cooperacion entre agentes y esta basado en el juego opensource Mari0. El \textbf{input} de este entorno son las posibles acciones que puede realizar un jugador en este videojuego. El \textbf{output} es una imagen en formato RGB de dimensiones 600 x 800. Adicionalmente, hemos generado la documentación necesaria para que otros investigadores puedan usar el entorno para entrenar agentes. Aparte, hemos conseguido que nuestro entorno siguiera la estructura de creación de entornos de OpenAI. Este entorno disponible unicamente en el sistema operativo de Linux.
    \item Hemos conseguido además implementar la funcionalidad de que pueda jugar un humano y varias inteligencias artificiales a la vez, pero siempre con la restricción de 4 jugadores como máximo.
    \item Finalmente hemos conseguido entrenar un par de agentes

\end{itemize}
 

