\section{Estructura del documento}

En esta sección presentaremos brevemente la estructura del documento. El objetivo es explicar brevemente cada apartado para guiar a los lectores a los apartados que más les interesen.

El primer capítulo del documento es la \nameref{section:Introducción}. En este capítulo introducimos los principales conceptos del aprendizaje por refuerzo, comentamos en que marco se sitúa el proyecto, su importancia, los objetivos que hemos planteado para el entorno a desarrollar y los posibles riesgos que pueda encontrar el trabajo.


El segundo capítulo del documento es el \nameref{section:Estado del arte}. En este capítulo repasamos exhaustivamente el estado del arte en entornos multiagente. En concreto capítulo esta divido en dos secciones, la primera consiste en entornos que ya pueden ser usados para el entrenamiento de agentes mientras que el segundo apartado está centrado en juegos y plataformas open source que puedan usarse para adaptar a este tipo de entornos. De cada entorno encontrado se analizan sus principales características.


En el capítulo \nameref{section:Propuesta y Metodología} planteamos la metodología que se utilizará para desarrollar el TFG. Además definiremos las métricas que se utilizaran para comparar los diferentes entornos encontrados en el apartado anterior. Finalmente se comparan los diferentes entornos y se selecciona el entorno final que se usará durante el resto del trabajo.



En el capítulo de \nameref{section:Desarrollo} nos centraremos en explicar las características del entorno. Adicionalmente comentaremos las tareas que fueron necesarias para adaptar el entorno a nuestros objetivos y discutiremos varias alternativas planteadas durante la realización del trabajo. Finalmente comentaremos el proceso de testing del entorno. Finalizando con la fase testing del entorno, en el capítulo \nameref{section:Entrenamiento de los agentes} explicaremos de forma resumida en las estrategias y algoritmos utilizados para el entrenamiento de agentes en el entorno desarrollado. Además de esto observaremos cuáles han sido los resultados empíricos del entrenamiento.



Finalmente a modo de conclusión de la parte técnica del trabajo en el capítulo \nameref{section:Resultados} comentaremos los resultados obtenidos en el trabajo, pasando por las características propias del entorno y los resultados obtenidos del entrenamiento de los agentes y la justificación de estos resultados. Como último capítulo están las \nameref{section:Conclusiones}. En este capítulo hablaremos sobre la consecución de los objetivos del trabajo, el posible trabajo futuro que se puede derivar de este, los conocimientos necesarios para desarrollar el trabajo así como los conocimientos adquiridos a en este.



En el \nameref{section:Anexos} se adjuntarán gran parte de los apartados relacionados con la gestión del proyecto. En concreto se comentarán las tecnologías utilizadas, la extensión prevista del proyecto en su inicio y la extensión real una vez finalizado, el presupuesto destinado a este en caso de ser un proyecto llevado a cabo por una empresa. También se explicarán temas de sostenibilidad y viabilidad del proyecto. Finalmente se analizará la normativa asociada a este TFG. En este apartado también están incluidos los agradecimientos y el glosario.