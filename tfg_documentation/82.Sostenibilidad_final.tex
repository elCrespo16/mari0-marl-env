\section{Sostenibilidad final}

Una vez finalizado el trabajo, es momento de valorar de nuevo los aspectos de la Sostenibilidad planteados en este apartado y como han sido tratados a lo largo de este proyecto. Para ello utilizaremos las preguntas proporcionadas por la asignatura de GEP para la evaluación de la sostenibilidad. En primer lugar nos platearemos la dimensión ambiental.

\subsection{Dimensión ambiental}

\textbf{¿Has cuantificado el impacto ambiental de la realización del proyecto?}

Puesto que no ha habido prácticamente ninguna diferencia en la planificación del proyecto además de un breve desbalance de las horas empleadas en este, podemos calcular el impacto ambiental del proyecto fácilmente. La cifra que calculamos de emisiones de CO2 en la sostenibilidad inicial era de 3556 kg de CO2. A estas emisiones es necesario añadirle las emisiones producidas por el uso energético de los ordenadores. Puesto que las horas de utilización de los ordenadores son 500 horas, el consumo de los ordenadores son 45 kWh y las emisiones producidas por un kWh son de 250 gCO2/kWh, podemos concluir que en consumo eléctrico hemos emitido 11.25 kg de CO2. Esto sumado con las emisiones iniciales da unas emisiones de 3568.25 kg de CO2.

\textbf{¿Qué medidas has tomado para reducir el impacto?}

Como ya habíamos mencionado, las medidas que hemos tomado para reducir el impacto consisten en usar ordenadores de segunda mano, con esto hemos conseguido amortizar el impacto de la creación de estos, además de escoger ordenadores eficientes energéticamente. No hemos podido implementar ninguna otra medida, ya que las horas de trabajo no se han podido reducir ni tampoco reducir el gasto de papel.

\textbf{¿Has cuantificado esta reducción?}

No ha sido posible cuantificar la amortización de haber usado ordenadores de segunda mano. Lo que si hemos podido cuantificar ha sido la eficiencia energética de los ordenadores. Estos tienen una potencia de 45 W mientras que un ordenador nuevo de las características que necesitábamos consume alrededor de 60 W. Esta medida nos ha reducido el impacto de CO2 en 5.62 kg aproximadamente.

\textbf{Si hicieras de nuevo el proyecto, ¿podrías realizarlo con menos recursos?}

No creemos que sea posible realizar este proyecto con menos recursos, ya que los recursos principales son imprescindibles. Si se reduce su uso entonces sería necesario eliminar características del proyecto.

\textbf{¿Qué recursos estimas que se usarán durante la vida útil del proyecto?}

Este proyecto está enfocado a otros investigadores de la rama de MARL. Los recursos que se usarán son los mismos que se han usado en este proyecto, ordenadores para el entrenamiento de agentes y puede que papel para los resultados de los experimentos. Y para el mantenimiento del entorno también se usarán ordenadores.

\textbf{¿Cuál será el coste ambiental de estos recursos?}

Es difícil estimar el coste ambiental, ya que no podemos saber que cantidad de material se podría utilizar. Lo que si sabemos es que el coste de crear un ordenador es 889 kg de CO2, y el coste por kWh es de 250 g CO2/kWh. A partir de estos costes podría calcularse el impacto ambiental de ellos. El impacto ambiental del posible mantenimiento del proyecto no es fácilmente estimable, pero podemos suponer un plazo de 150 horas de mantenimiento. Por lo tanto el coste ambiental del mantenimiento sería de alrededor de 1.12 kg de CO2.

\textbf{¿El proyecto permitirá reducir el uso de otros recursos?}

Este proyecto puede permitir que se reduzcan el uso de otros recursos. Esto se debe a que gracias a este proyecto pueden investigarse tecnologías nuevas que pueden llevar a soluciones tecnológicas más eficientes. Aun así, es difícil cuantificar esta posible reducción, ya que se trata de un campo muy nuevo y del que todavía no se conoce su potencial.

\textbf{¿Globalmente, el uso del proyecto mejorará o empeorará la huella ecológica?}

El uso de este proyecto mejorará la huella ecológica, ya que otros investigadores no deberán destinar recursos a la creación de un entorno de este estilo.

\textbf{¿Podrían producirse escenarios que hiciesen aumentar la huella ecológica del proyecto?}

No existen muchos escenarios que pudieran aumentar la huella ecológica del proyecto, ya que en sí el proyecto no depende de otros elementos. La huella ecológica del uso del proyecto solo depende del ordenador en el cual se utilice, por lo tanto no podrían producirse escenarios que pudieran aumentar la huella ecológica del proyecto.

\subsection{Dimensión económica}

\textbf{¿Has cuantificado el coste (recursos humanos y materiales) de la realización del proyecto?}

El coste de la realización del proyecto es el coste que presupuestamos en el apartado económico del proyecto. Aunque el final del proyecto se aplazará una semana aproximadamente, esto sé previó en el presupuesto, de forma que no afectó al coste de este. Aun así, en este presupuesto no se tuvo en cuenta el posible valor del mantenimiento. En el apartado anterior hemos supuesto unas 150 horas de trabajo, por lo tanto sería necesario añadir alrededor de 150 * 26 = 3900 € para el mantenimiento del proyecto.

\textbf{¿Qué decisiones has tomado para reducir el coste?}

Para reducir el coste del proyecto utilizamos ordenadores de segunda mano, que tienen un coste mucho menor al coste de uno nuevo. Además de esto, no fue posible implementar ninguna otra medida para reducir su coste, ya que solo se consideró usar los recursos mínimos desde el principio.

\textbf{¿Has cuantificado este ahorro?}

En caso de no haber utilizado un ordenador de segunda mano y haber comprado nuevos, el precio por cada uno sería de aproximadamente 800 € por ordenador y 1000 € por el servidor. Esto hace que el ahorro total por usar ordenadores de segunda mano sea de 300 * 3 + 200 = 1100 € aproximadamente.

\textbf{¿Qué coste estimas que tendrá el proyecto durante su vida útil?}

Como hemos comentado anteriormente, el coste del proyecto es el coste inicial de este que son 15710.9 € más el coste del mantenimiento de este. Suponiendo 150 horas de trabajo del programador del equipo, que tiene un precio de 26 €/h, el coste del mantenimiento es de 3900 €. El coste total del proyecto durante su vida útil estimamos que será aproximadamente de 19610.9 €.

\textbf{¿Se podría reducir este coste para hacerlo más viable?}

Este coste no se puede reducir, ya que no se puede reducir el tiempo de mantenimiento. Si se redujera este tiempo, algunas tareas no podrían realizarse.

\textbf{¿Se ha tenido en cuenta el coste de los ajustes/actualizaciones/reparaciones durante la vida útil del proyecto?}

En este último apartado se ha tenido en cuenta el mantenimiento del proyecto durante la vida útil del proyecto. Este coste de mantenimiento incluye los posibles ajustes o actualizaciones que se deban añadir. Aun así, no se planea añadir directamente nuevas características al entorno, solo tareas de mantenimiento.

\textbf{¿Podrían producirse escenarios que perjudicasen la vida útil del proyecto?}

Los escenarios que se podrían producir que afectasen a la vida útil del proyecto son que las tecnologías de las cuales este depende se volvieran obsoletas o se eliminasen completamente. Esto afectaría de forma que el entorno ya no podría ser usado y habría que buscar una alternativa para substituir el componente que ha sido eliminado. Esto afectaría negativamente al coste, puesto que habría que invertir muchos recursos para arreglar el entorno.

\subsection{Dimensión social}

\textbf{¿La realización de este proyecto ha implicado reflexiones significativas a nivel personal, profesional o ético de las personas que han intervenido?}

La principal reflexión que ha implicado este proyecto ha sido la importancia del software libre. Gracias a las personas que desarrollan este tipo de software muchos otros proyectos son posibles. Por eso considero que estos proyectos deberían ser más apoyados dentro de la comunidad tecnológica.

\textbf{¿Quién se beneficiará del uso del proyecto?}

Los principales beneficiados del uso del proyecto son la comunidad de MARL, ya que esta comunidad es quien usará esta solución principalmente. Además, de forma indirecta, también se beneficiarán todas las personas a las cuales la comunidad de RL aporte alguna mejora.

\textbf{¿Hay algún colectivo que puede verse perjudicado por el proyecto?}

En principio no hay ningún colectivo que pueda verse directamente perjudicado por el proyecto. Aun así, las clases de menor capacidad adquisitiva probablemente no tendrían acceso a la utilización de este proyecto, ya que son necesarios unos requisitos mínimos de hardware.

\textbf{¿En qué medida?}

El perjuicio de este proyecto sería mínimo, puesto que solo implicaría que las clases de menor capacidad adquisitiva no podrían tener acceso a esta solución. Pero esto de forma general no afectaría de forma negativa a su estilo o calidad de vida.

\textbf{¿En qué medida soluciona el proyecto el problema planteado inicialmente?}

El problema planteado inicialmente era la falta de entornos con aspectos sociales en el entrenamiento de varios agentes. Aunque un solo entorno nuevo no soluciona el problema por completo, aporta un poco más de variedad a los entornos existentes.

\textbf{¿Podrían producirse escenarios que hiciesen  que el proyecto fuese perjudicial para algún segmento particular de la población?}

No podrían producirse escenarios que hiciesen que el proyecto fuese perjudicial para algún segmento particular de la población. Este proyecto solo se trata de un entorno para entrenar inteligencias artificiales, así que no hay ningún escenario realista en el cual esto pueda afectar a algún segmento de la población.

\textbf{¿Podría crear el proyecto algún tipo de dependencia que dejase a los usuarios en posición de debilidad?}

Puesto que existen otros entornos similares, este proyecto no debería crear una situación de dependencia en los usuarios. Siempre existe la posibilidad de usar otro entorno similar.

