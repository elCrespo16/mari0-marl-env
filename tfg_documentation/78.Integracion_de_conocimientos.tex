\section{Integración de conocimientos}

Para la realización de este trabajo se han integrado los conocimientos de varias asignaturas estudiadas en la carrera y estos conocimientos han hecho posible la consecución de los objetivos propuestos.

\begin{itemize}
    \item Aprendizaje automático (APA) y Minería de Datos (MD): gran parte de los conocimientos aprendidos en estas asignaturas han servido como base para entender los algoritmos usados en RL. Como por ejemplo el uso de redes neuronales para el aprendizaje.
    \item Sistemas operativos (SO): La parte de la comunicación entre procesos ejecutada en el proyecto ha sido posible gracias a esta asignatura.
    \item Paralelismo (PAR): La coordinación y comunicación entre threads no habría sido posible sin los conocimientos aprendidos en esta asignatura.
    \item Arquitectura del software (AS): Muchos de los patrones de diseño usados en el proyecto han sido aprendidos gracias a esta asignatura. Sin ellos el código sería mucho menos legible y más complicado de modificar.
    \item Lenguajes de programación (LP): Esta asignatura aportó los conocimientos necesarios para aprender de forma más rápido los conceptos de \textit{Lua} y \textit{Python}.
\end{itemize}

Además de los conocimientos aprendidos en estas asignaturas, muchos otros conocimientos tuvieron que ser adquiridos a lo largo del trabajo de forma autónoma. Principalmente los conocimientos de las principales técnicas de Reinforcement Learning como DQN, CNN, etc.

