\chapter{Conclusiones}

\label{section:Conclusiones}

Gracias a las fases de planificación y presupuesto hemos conseguido completar todos los objetivos propuestos en este proyecto. En primer lugar, hemos conseguido realizar una búsqueda en el estado del arte de los entornos multiagente y posibles plataformas open source que pudieran ser adaptadas para convertirse en este tipo de entornos para MARL (OBJ 1). Esto puede verse en el apartado de estado del arte. En segundo lugar, hemos conseguido desarrollar unas métricas cualitativas para poder comparar los diferentes entornos que habíamos encontrado previamente (OBJ 2). Esto puede verse en apartado de métricas para los entornos. Adicionalmente hemos garantizado que el entorno usa la estructura de entornos de \textit{PettingZoo} (OBJ 3). Vemos las operaciones que teníamos que implementar en el apartado de entornos \textit{PettingZoo}. 

También hemos completado nuestro objetivo principal, crear un entorno multiagente de como máximo 4 jugadores que obliga a la cooperación entre estos. Este entorno dispone de una documentación extensa. Adicionalmente permite uno de los jugadores del entorno sea un jugador humano, cosa que permite aplicar otros métodos de entrenamiento más variados. Finalmente, conseguimos entrenar agentes en este entorno, confirmando que es un entorno viable para el entrenamiento (OBJ 4). Este proceso ha sido detallado en el apartado de entrenamiento de los agentes. Desafortunadamente no se ha conseguido que estos agentes solucionen por completo todo el entorno.


Durante la realización de este entorno se ha aprendido de diversas tecnologías como \textit{Xvfb}, \textit{VNC} y sobre todo sobre el campo de Reinforcement Learning. También se han utilizado capacidades no técnicas como planificación y gestión de proyectos, resolución de problemas, búsqueda de información, etc.

Una vez concluido la realización del proyecto, trataremos de que este entorno sea incluido en las librerías de \textit{PettingZoo} y de \textit{OpenAI}. Ya hemos visto que el entorno sigue la estructura de creación de entornos de \textit{PettingZoo}, pero además, al ser posible jugar un solo jugador, este entorno también puede ser utilizado para \textit{OpenAI} en Single Agent Reinforcement Learning. Añadir este proyecto a estas librerías ayudaría a que fuera más reconocido y otros investigadores pudieran utilizarlo más fácilmente.