\chapter{Conclusiones}

Gracias a las fases de planificación y presupuesto hemos conseguido completar todos los objetivos propuestos en este proyecto. En primer lugar, hemos conseguido realizar una búsqueda en el estado del arte de los entornos multiagente y posibles plataformas open source que pudieran ser adaptadas para convertirse en este tipo de entornos para MARL. En segundo lugar, hemos conseguido desarrollar un entorno multiagente de como máximo 4 jugadores que obliga a la cooperación entre estos. Este entorno sigue el patrón de entornos de PettingZoo y dispone de una documentación extensa. Adicionalmente permite uno de los jugadores del entorno sea un jugador humano, cosa que permite aplicar otros métodos de entrenamiento más variados. Finalmente, conseguimos entrenar agentes en este entorno, confirmando que es un entorno viable para el entrenamiento. Desafortunadamente no se ha conseguido que estos agentes solucionen por completo todo el entorno.


Durante la realización de este entorno se ha aprendido de diversas tecnologías como Xvfb, VNC y sobre todo sobre el campo de Reinforcement Learning. También se han utilizado capacidades no técnicas como planificación y gestión de proyectos, resolución de problemas, búsqueda de información, etc.