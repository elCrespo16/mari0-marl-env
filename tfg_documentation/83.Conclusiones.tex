\section{Conclusiones}

Gracias a las fases de planificación y presupuesto hemos conseguido completar todos los objetivos propuestos en este proyecto. En primer lugar, hemos conseguido realizar una busqueda en el estado del arte de los entornos multiagente y posibles plataformas opensource que pudieran ser adapatadas para convertirse en este tipo de entornos. En segundo lugar, hemos conseguido desarrollar un entorno multiagente de como maximo 4 jugadores que obliga a la cooperacion entre estos. Este entorno sigue el patron de entornos de PettingZoo y dispone de una documentación extensa. Adicionalmente permite uno de los jugadores del entorno sea un jugador humano, cosa que permite aplicar otros metodos de entrenamiento más variados. Finalmente, conseguimos entrenar agentes en este entorno ::::::::::::::::::::::


Durante la realización de este entornos se ha aprendido de diversas tecnologias como Xvfb, VNC y sobretodo sobre el campo de Reinforcement Learning. Tambien se han utilizado capacidades no tecnicas como planificación y gestion de proyectos, resolución de problemas, busqueda de información, etc.