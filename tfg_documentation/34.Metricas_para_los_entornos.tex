\section{Métricas para los entornos}
Como habíamos expuesto anteriormente en la sección de la planificación de los objetivos, una de nuestras metas consistía en crear unas métricas con las cuales podríamos valorar los entornos que nos encontráramos y a partir de esto escoger cuál de todos se adaptaría mejor a nuestro objetivo de crear un entorno multiagente con aspectos sociales complejos. En este apartado explicaremos cuáles serán los criterios que usaremos para filtrar los entornos y decidir sobre ellos. Ahora presentaremos cuáles son los criterios y porque son importantes y en el siguiente apartado clasificaremos los entornos que encontremos según estos criterios. En primer lugar, debemos tener en cuenta que trataremos dos tipos de entornos, los entornos que ya son sirven para entrenar agentes en sí y los entornos que es necesario adaptarlos para entrenar agentes. Es importante tener clara esta diferencia, puesto que los criterios pueden variar en función de si es de una categoría u otra. Por lo tanto, uno de los criterios también será de que tipo de entorno se trata.  

Los criterios que se evaluaran de cada entorno serán los siguientes:
\begin{itemize}
    \item \textbf{¿La documentación del entorno es buena? (DOC)}: Esto es precisamente importante en los proyectos grandes porque es fácil perderse intentando entender código escrito por otras personas.
    \item \textbf{¿Se trata de un entorno que ya sirve para entrenar de por si o es necesario adaptarlo?}: Tal y como hemos explicado antes esto es importante para decidir los criterios, además de que si el entorno es de un tipo u otro esto nos obligará a realizar acciones distintas. Si ya sirve para entrenar entonces deberemos añadir aspectos sociales que no disponga para que sea interesante. En cambio si se tiene que adaptar, escogeremos uno que ya tenga estos aspectos implementados y lo adaptaremos para que agentes puedan entrenar en él.
    \item \textbf{¿Dispone de mecanismos de negociación?(NEG)}: Este es uno de los aspectos sociales que pueden ser interesantes implementar, ya que son muy pocos los entornos que lo implementan y es interesante ver que clase de estrategias pueden crear los agentes para negociar con otros y conseguir sus objetivos.
    \item \textbf{¿Se obliga a cooperar?(COOP)}: Otro de los aspectos sociales que serían interesantes añadir sería la cooperación entre agentes. Puesto que la cooperación puede venir de muchas formas distintas es interesante que estrategias se pueden desarrollar dependiendo de las posibilidades que ofrece el entorno.
    \item \textbf{¿Se obliga a competir?(VS)}: Este criterio al igual que el anterior es interesante para ver que tipo de estrategias se pueden desarrollar agentes que compiten con otros. Al igual que la cooperación, la competición puede tener muchas formas distintas y las estrategias pueden ser diferentes entre cada una de ellas.
    \item \textbf{¿Permite la comunicación estructurada?(COM-E)}: Con comunicación estructurada nos referimos a mensajes predefinidos que puede enviar el agente. Esta definición también incluye emojis o stickers tan habituales en los videojuegos. Aunque parezca que esto no tiene ninguna relevancia, estos mensajes predefinidos podrían ser usados por los agentes para comunicarse entre ellos dándoles sus propios significados.
    \item \textbf{¿Permite la comunicación por lenguaje natural?(COM-N)}: También es muy común que en algunos entornos sea posible el uso de un chat donde se pueda usar lenguaje natural para comunicarse. Esto también podría ser interesante para el estudio de que uso pueden darle los agentes a esta herramienta.
    \item \textbf{¿Existen restricciones que puedan romperse dentro del juego?(RES)}: Otro aspecto que puede ser interesante estudiar es observar que estrategias escogerán los agentes si tienen la capacidad de romper algunas restricciones del entorno que les provoque un perjuicio con el objetivo de conseguir un beneficio distinto. Un ejemplo de esta mecánica puede verse en el fútbol con el sistema de las faltas. Las faltas teóricamente son penalizaciones por comportamientos que no deberían realizarse, pero en la práctica se realizan faltas en diversas situaciones porque pueden resultar beneficioso, aunque de por sí realizar una falta trae consigo una penalización.
\end{itemize}

Además, si se trata de un entorno que ya sirve para entrenar agentes los criterios serán los siguientes:
\begin{itemize}
    \item \textbf{¿Sigue la filosofía de construcción de entornos de Gym de OpenIA?(GYM}: Esto es importante, ya que si un entorno sigue esta filosofía, los elementos a implementar están claros y bien definidos, haciendo que el código sea más sencillo de entender y fácil de manipular. 
    \item \textbf{¿No está limitado a un cierto numero de agentes?(LIM)}: Este criterio puede ser importante, debido a que en algunos casos el número de agentes que se pueden entrenar al mismo tiempo puede ser significativo en una investigación.
\end{itemize}

Finalmente si se trata de un entorno no funcional tendremos en cuenta también lenguaje o framework se usa para desarrollar este entorno y la complejidad del entorno.



