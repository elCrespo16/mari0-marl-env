\section{Trabajo futuro}

En primera instancia, el entorno solo se penso para que pudieran jugar 2 agentes y así se hizo. Pero una vez acabada esta fase, se planteo añadir que en vez de ser como maximo 2 jugadores fueran 4. Adicionalmente tambien se planteo que se pudiera jugar un humano además de inteligencias artificiales. Y puesto que gracias al diseño del entorno implementar estas funcionalidades era relativamente trivial, se llevo a practica y se consiguio.

Aún así, planteamos posibles mejoras que se podrían añadir al entorno para mejorarlo. Esta ideas no se han podido implementar por falta de tiempo, de conocimientos o simplemente porque quedaban fuera del alcance del TFG. Las ideas que proponemos son las siguientes:

\begin{itemize}
    \item Creación de nuevos mapas con mayor complejidad o que requieran de más de dos jugadores para resolverse.
    \item Usar Docker para que el entorno no sea exclusivo de Linux y pueda usarse en cualquier sistema operativo.
    \item Creación de mapas que obliguen a la competición en vez de a la cooperación.
    \item Creacion de agentes capaces de solucionar el entorno.
\end{itemize}

Estas ideas podrían ser parte de otro trabajo de fin de grado además de aportar más valor al entorno y al campo de MARL.